\documentclass[a4paper]{article}
\title{Introduction to Cryptography} 
\author{Angelo Panariti}
\begin{document}

\maketitle
\centerofcontents
\section{The modulo operator}
Let $a, r, m \in Z$ and $m > 0$, we write $a \equiv mod \ m$.
If $m$ divides $(a -r)$ $\implies$ 

$$m | (a - r)$$

\subsection{Exercise}

$$a = 13, m = 9, r=?$$
$$13 \equiv 4 \ mod \ 9$$
$$a-r = (13 - 4) = 9$$
\subsection{Computation of the remainder}

given: $a, m \in Z$
$$ a = qm $$
\subsubsection{Example}

$$ a = 42, m = 9$$
$$42 = 4 \cdot 9 + 6 \rightarrow 6 \equiv r = 6$$
Check ($42-6) = 36, 9/36$

$$42 = 3 \cdot 9 + 15 \rightarrow r = 15$$

Check $(42 - 15) = 27, 9/27$
 
$$42 = 5 \cdot 9 + ( 3 \cdot 3) \rightarrow r = 3$$

Check $(42 - 3) - 39, 9/39$. 

The remainder is not unique.
\subsection{Equivalence Classes}
Ex: $a=12$, $m=5$
$$12 \equiv 2 mod 5$$
$$12 \equiv 7$$
\subsubsection{Definition}
{The set $\{...,-8,-3, 2, 7, 12 ,17,...\}$ forms an equivalence class modulo 5. All members of the class behave equivalent modulo 5. Let's look at all the equivalence classes modulo 5.}. 

$$A = \{...,-10, -5, 0, 5, 10,...\}$$
$$B = \{..., -9, -4, 1, 6, 11, 16,...\}$$
$$C =\{...,-8,-3,2,7,12,...\}$$
$$D =\{...,-7,-2,3, 8, 13,...\}$$



$13\cdot16-8=200\equiv0mod5$

\

If we replace every number of this expression with the respective number in the equivalence class, we get:

\


$ 3\cdot1=3-3\equiv0mod5$

\

$8\cdot6-(-7)=48+7=55\equiv0mod5$

\subsection{Important application}

Ex: $$3^8 \ mod \ 7 \equiv \ ?$$

\subsubsection{First way}
$$3^8 = 6561 \equiv \ 2 $$

\subsubsection{Second way}
$$3^8 = 3^4 \cdot 3^4 = \ 81 \cdot 81 \equiv 4\cdot4=16\equiv \ 2 \ mod \ 7 $$

\section{Rings: An Algebraic View on Modular Arithmetic}

\subsection{Definition}
The integer ring $Z_m$ consists of-

\begin{enumerate}
\item
   $Z_m = \{ 0, 1, 2, \ldots, m-1\}$
\item
    Two operators "+" and "\cdot" so that for all $a,b,c \in Z_m$
    \begin{enumerate}
        \item $a+b \equiv c \ mod \ m$
        \item $a\cdot b \equiv \ d \ mod \ m$
    \end{enumerate}

\end{enumerate}
\subsection{Example for multiplicative inverses}
$$m = q$$

$$2\cdot 2^{-1} \equiv 1 \ mod \ 9$$
$$2\cdot 5 \equiv 1 \ mod \ 9$$
$$2^{-1} \equiv 5 \ mod \ 9$$

\section{Shift (or Caesar) Cipher}
Idea: shift every letter in the alphabet by a certain number of letters. For example, let's take a variable $k = 3$ (an arbitrary number). So for instance, $A$ is replaced with $D$, $B$ is replaced with $E$, $X$ is replaced with $A$ (at the end of the alphabet, we just just "wrap around" and start at the beginning again), et cetera (modulo 26).

\
\subsection{Shift Cipher}
Let $x,y,k \in Z_26$

\

Encryption: $e_k(x) = x+y \ mod \ 26$

Decryption: $d_k(y) = x+k \ mod \ 26$



\subsection{Affine Cipher}
$$k=(a,b)$$
$$y \equiv a \cdot x +b \ mod \ 26$$
$$y-b \equiv a \cdot x$$

\subsection{Two attacks possible}
\begin{enumerate}
\item {frequency analysis}
\item {brute-force attack}
\end{enumerate}

\section{Stream Ciphers}
Streaming bits, flowing one after another. A stream cipher encrypts bits individually and its equation is pretty similar to the Shift Cipher one:

\

Encryption: $$y_i = e(x_i) \equiv x_i +s_i \ mod 2$$
Decryption: $$x_i = d(y_i) \equiv y_i -s_i \ mod 2$$

Question: Why are decryption and encryption the same operation?

$$d(y_i) \equiv y_i -s_i \ mod 2$$
$$ \equiv (x_i + s_i) + s_i \ mod 2$$
$$ \equiv x_i + 2s_i \ mod 2$$
$$ \equiv x_i \ mod 2$$


$$x_i, y_i, z_i \in Z = \{ 0, 1\}$$

$mod \ 2$ addition and subtraction are the same operation.

Closer look at the mod2 add:


\begin{center}
    \begin{tabular}{c c c c}
        $x$ & $s_i$ & $y_i$ mod 2 \\
        0 & 0 & 0 \\
        0 & 1 & 1 \\
        1 & 0 & 1 \\
        1 & 1 & 0
\end{tabular}
\end{center}

mod2 addition is the same as the XOR operation.

Example: encryption of ASCII letter "A"

$$x_7...x_1=1000001$$
$$s_7...s_1=0101101$$
$$s_7...s_1=1101100$$

Question: how to generate the key stream bits $s_i$?
Somehow related to randomness!

\section{Random numbers generators (RNG)}

We distinguish between three types of RNGs:
\begin{enumerate}
\item
    True random numbers generators (TRNG)
        \begin{itemize}
    \item True random numbers stem from physical processes, e.g: coin-flipping, dice rolls, etc.
  \end{itemize}
\item
    Pseudo random number generators (PRNG)
    \begin{itemize}
        \item  PRNs are computed, i.e, they are deterministic. (if you can compute something, it's not really random). Often they are computed with the following function, namely: $s_0 = seed$, each subsequent $s_i+1=f(s_i)$, recursively being computed.
    \item Ex: rand() function in ANSI C, it has a fixed $s_0 = 123456$, s_{i+1}=110 3515245 s_i + 12345 \ mod \ 2^31 
    \end{itemize}
\item
    Cryptographically secure PRNGs (CSPRNG)
    \begin{itemize}
        \item
            CPRNGs are PRNs, but they are not deterministic. They are not computed deterministically, but rather based on the hardware's random number generator. They are therefore not truly random. So they are computed with an additional property: the numbers are unpredictable 
        \item
            Informal definition: given $n$ output bits, $s_i, s_i + 1, ..., s_{i+n-1}$, it is computationally infeasible to construct $s_n$
    \end{itemize}
\end{enumerate}

\section{One Time Pad (OTP)}
\begin{itemize}
    \item Goal: build a "perfect" encryption algorithm.
    \item Definition: A cipher is unconditionally secure (or information theoretically secure) if it cannot be broken by infinite computing resources.
    \item Definition: The One Time Pad (OTP) is a stream cipher where:
        \begin{enumerate}
        \item
            The first stream bits $s_i$ stem from a TRNG
        \item
            Each key stream bit is used only once
        \end{enumerate}
        \begin{itemize}
            \item
                Big drawback: key is as long as the message. Ex: encryption of a $400MB$ file $\rightarrow 8 \cdot 400MB = 3.2GBit$ of key
        \end{itemize}
\end{itemize}

\section{Linear Congruential Generator (LCG)}

\begin{itemize}
    \item
        Idea: Use a key-stream $s_i$ from a PRNG$.
\end{itemize}
        $$S_g = seed $$
        $$S_{i+1} = A \cdot S_i + B \ mod \ m, A,B,S_i, \in  Z_m$$
        key $ k=(A,B)$. Remark: $A,B,S_i$ are [$log_2m$] bits long.
       
        Attack example: Oscar knows: x_1, x_2, x_3.
        1) Oscar computes S_1, S_2, S_3

        $$S_2 \equiv A^1 \cdot S_1 + B^1 \ mod \ m$$
        $$S_2 \equiv A \cdot S_1 + B \ mod \ m$$

        Linear equation system with 2 unknowns: 

        $$ A = (S_2 - S_3) \cdot (S_1 - S_2)^-1 \ mod \ m$$
        $$ B = S_2 - S_1 (S_2 - S_3) (S_1 - S_2)^-1 \ mod \ m$$

\end{document}


